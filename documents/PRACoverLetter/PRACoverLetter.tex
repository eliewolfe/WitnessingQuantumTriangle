\documentclass[12pt]{letter}
\usepackage[utf8]{inputenc}
\usepackage[scale=0.85]{geometry}
\usepackage{lmodern}
\begin{document}
\begin{letter}{Physical Review A (PRA)}
{\fontfamily{lmss}\selectfont
\opening{Dear Editorial Board and Journal Staff,}

We hereby submit our manuscript, “Causal Compatibility Inequalities Admitting of Quantum Violations in the Triangle Structure”.

A necessary ingredient of any manifestly quantum technology is the capitalization of some quantum phenomenon which cannot be emulated by classical resources. Consequently, a foundational objective of quantum information theory is the characterization of situations wherein quantum resources offer unique advantages. Our manuscript concerns the exceptionally paradigmatic ability of quantum theory to violate certain constraints which are satisfied by all classical correlations.

The dual practices of deriving classically-inviolable inequality constraints, and quantifying their violation by quantum resources, have proven enormously effective in the development of quantum computational algorithms and quantum communication protocols. The first situation studied in this manner consisted of two distant parties each making measurements on their respective portions of a shared bipartite state; the pertinent inequalities there being the famous Bell inequalities. This specific type of cataloging of quantumness – of which the violation of Bell inequalities is emblematic – has since been extended to other situations involving multiple parties, sharing multiple states in a variety of configurations.

However, a particular configuration of state-sharing, technically-termed “cyclic”, has proven heretofore intractable. The simplest cyclic situation is the “Triangle” structure, wherein three parties each make measurements on respective portions of three bipartite states. Although a variety of classically-inviolable inequalities for the Triangle structure have been derived in the last few years, none of those initial inequalities could be shown to admit violation by quantum resources. Identifying a quantumly-violable yet classically-inviolable inequality for the Triangle structure has persisted as a longstanding and vexing open problem, despite numerous attempts by other authors.

Our submission provides resolution to this problem. We derive several classically-inviolable inequalities for the Triangle structure which are demonstrably violated by quantum resources. We discuss three qualitatively-distinct inequalities in particular. A consequence of our derivation is that – for the first time – it is now possible to robustly verify manifestly quantum behaviors in the Triangle structure. As such, our submission opens the door for the Triangle structure to be capitalized in quantum technologies.

Moreover, the pursuit of our results also sheds informative light on lesser-known mathematical methods.  For instance, in deriving our inequalities we utilized the so-called “Inflation Technique” for solving general causal inference problems. The Inflation Technique was developed for the explicit purpose of identifying manifestly quantum behaviors; our manuscript is the first practical validation that the technique is indeed capable of doing so.

In consideration of the aforementioned achievements and insights contributed by our manuscript, we are confident that this submission meets PRA’s high standards of substantiveness and significance. In preparing this submission for PRA, we have taken care to write the text in a deliberate style that is both clear and comprehensive.

Sincerely,
Thomas C. Fraser and Elie Wolfe
}
\end{letter}

\end{document}