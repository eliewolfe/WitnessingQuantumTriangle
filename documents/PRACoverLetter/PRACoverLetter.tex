\documentclass[12pt]{letter}
\usepackage[utf8]{inputenc}
\usepackage[scale=0.85]{geometry}
\usepackage{lmodern}
\begin{document}
\begin{letter}{Physical Review A (PRA)}
{\fontfamily{lmss}\selectfont
\opening{Dear Editorial Board and Journal Staff,}

We hereby submit our manuscript, "Causal Compatibility Inequalities Admitting of Quantum Violations in the Triangle Structure".

% Paragraph 1: Why look for non-classical, quantum correlations?

A necessary ingredient of any manifestly quantum technology is the exploitation of some quantum phenomena which fails to be emulated by classical resources. Inherently, the advantage of quantum resources is their ability to violate correlation constraints satisfied by all classical correlations. A primal objective of quantum information theory and quantum foundations is to discover new and practical situations wherein quantum theory offers an advantage and to quantify that advantage via inequality constraints. Developing tools which solve this foundational problem in turn enhance out ability to assess the limitations and capabilities of quantum computational algorithms and quantum communication protocols.

% Paragraph 2: The Triangle Structure and the long-standing unsolved problem.

Arguably the most prominent scenario in which such inequality constraints have been derived is the Bell structure, for which the inequalities are known as Bell inequalities. The Bell structure consists of two distant parties each making measurements on their respective portions of a shared bipartite state. The focus of the present submission is to progress our understanding of analogous constraints for a superficially simple structure called the Triangle structure wherein three parties arrange themselves in a triangle and each make measurements on respective portions of three bipartite states. Despite numerous attempts by other authors, finding inequalities constraining classical correlations for the Triangle structure which admit violations by quantum resources as heretofore remained a long-standing open problem.

% Paragraph 3: Achievements of this submission.

This submission provides a resolution to this problem, by reporting the derivation of several polynomial inequalities for the Triangle structure which, importantly, are violated by quantum resources. Three qualitatively distinct inequalities are reported in the document for discussion. These inequalities constitute the first verification of genuine non-classical behaviour in the Triangle structure, promoting its exploitation for quantum technologies. Moreover, we demonstrate that these inequalities are suitable measures of non-classicality for experimental correlations as they are robust to a suitable degree of noise.

% Paragraph 4: Future outlook.

Perhaps of chief importance, new methods and tools where developed in pursuit of obtaining these results. In particular, these inequalities were derived using a novel technique for solving general causal inference problems known as the Inflation Technique. This submission demonstrates that the Inflation Technique has leading utility for studying and identifying genuine quantum correlations in more involved scenarios and consequently, in the discovery of new quantum resources.

In consideration of the aforementioned achievements and insights, we are confident this submission meets PRA's standards.

Sincerely,

Thomas C. Fraser and Elie Wolfe
}
\end{letter}

\end{document}