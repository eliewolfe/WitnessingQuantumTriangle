% === Revtex Declaration ===
\documentclass[aps, 10pt, english, twoside, pra, nofootinbib, longbibliography]{revtex4-1}

% === All of the Packages I use frequently ===
\usepackage{../../packages/document_config}
\usepackage{../../packages/shared}
\usepackage{../../packages/misc_commands}

% === Causal structure formatting for tikz ===
% =========================
% Causal Structure Diagrams
% =========================
\definecolor{obs_outline}{RGB}{51,157,215}
\definecolor{obs_fill}{RGB}{222,253,255}
\definecolor{obs_text}{RGB}{0,0,0}
\definecolor{lat_outline}{RGB}{251,141,54}
\definecolor{cause}{RGB}{30, 0, 30}
\definecolor{lat_fill}{RGB}{255,213,153}
\definecolor{lat_text}{RGB}{0,0,0}
\tikzset{square/.style={regular polygon,regular polygon sides=4}}
\tikzset{triangle/.style={regular polygon,regular polygon sides=3}}
\tikzset{observed/.style={obs_text, align=center, triangle, thick, draw=obs_outline, fill=obs_fill, inner sep=-0.2em, text width=1.5em}}
\tikzset{latent/.style={lat_text, align=center, circle, thick, draw=lat_outline, fill=lat_fill, text width=1.5em, inner sep=0.2em}}
\tikzset{fade/.style={opacity=0.2}}
\tikzset{unfade/.style={opacity=1.0}}
% TikZ stile to apply keys only on specific beamer overlays
% onslide=<overlay spec>{key=value, key=value, ...}
\tikzset{onslide/.code args={<#1>#2}{%
  \only<#1>{\pgfkeysalso{#2}}%
}}
\providecommand{\p}[1]{#1}
% \tikzset{cause/.style={mid arrow/.style={postaction={decorate,decoration={markings, mark=at position .5 with {\arrow[#1]{stealth}}}}},}}
\tikzset{
    % style to apply some styles to each segment of a path
    on each segment/.style={
        decorate,
        decoration={
            show path construction,
            moveto code={},
            lineto code={
                \path [#1]
                (\tikzinputsegmentfirst) -- (\tikzinputsegmentlast);
            },
            curveto code={
                \path [#1] (\tikzinputsegmentfirst)
                .. controls
                (\tikzinputsegmentsupporta) and (\tikzinputsegmentsupportb)
                ..
                (\tikzinputsegmentlast);
            },
            closepath code={
                \path [#1]
                (\tikzinputsegmentfirst) -- (\tikzinputsegmentlast);
            },
        },
    },
    % style to add an arrow in the middle of a path
    mid arrow/.style={postaction={decorate,decoration={
                markings,
                mark=at position .6 with {\arrow[scale=1.5, cause]{stealth}}
            }}},
}
% =========================
% =========================

\begin{document}
    \title{Causal Compatability Inequalities Admitting of Quantum Violations in the Triangle Scenario}
    \author{Thomas C. Fraser}
    \email{tcfraser@tcfraser.com}
    \affiliation{Perimeter Institute for Theoretical Physics, Waterloo, Ontario, Canada \\ University of Waterloo, Waterloo, Ontario, Canada}
    % \author{Elie Wolfe}
    % \email{ewolfe@perimeter@institute.ca}
    % \affiliation{Perimeter Institute for Theoretical Physics, Waterloo, Ontario, Canada}
    \date{\today}
    \begin{abstract}
        Quantum correlations are often incompatible with a classical assumption of causal structure. This nonclassicality is often known as quantum nonlocality, and it is witnessed through the violation of causal compatability inequalities, such as Bell inequalities. Such inequalities were recently derived for the Triangle scenario [arXiv:1609.00672], begging the question: can these inequalities be violated by quantum correlations? Here we answer this affirmatively, and discuss specific Triangle scenario inequalities and quantum configurations which manifest nonclassical correlations. Numerical optimzations reveal quantum resources potentially qualitatively different from those known previously.
    \end{abstract}
    \maketitle
    \tableofcontents

    \section{Introduction}
    \begin{itemize}
        \item \todo[TC]{Overview of importance of inequalities}
        \item \todo[TC]{Triangle Scenario and existing work}
        \item \todo[TC]{Objective of research project}
        \item \todo[TC]{Structure of this paper}
    \end{itemize}

    \section{Causal Compatibility}
    \begin{itemize}
        \item \todo[TC]{Define marginal scenario}
        \item \todo[TC]{Define marginal model}
        \item \todo[TC]{Define causal structure}
        \item \todo[TC]{Define compatibility}
    \end{itemize}

    \section{Triangle Scenario}
    \begin{itemize}
        \item \todo[TC]{Discuss some of its appearances in other work}
        \item \todo[TC]{Figure}
    \end{itemize}
    \subsection{Fritz Distribution}
    \begin{itemize}
        \item \todo[TC]{Reproduce Fritz Distribution}
        \item \todo[TC]{Recall its proof of incompatibility}
        \item \todo[TC]{Requires perfect correlations}
        \item \todo[TC]{Not robust to noise}
        \item \todo[TC]{Fritz Problem 2.17}
    \end{itemize}
    \section{Inflation Technique}
    \begin{itemize}
        \item \todo[TC]{Summarize inflation technique}
        \item \todo[TC]{Inflations of Triangle Scenario}
        \item \todo[TC]{Demonstrate that one can derive causal incompatibility inequalities from inflation}
        \item \todo[TC]{Pre-injectable sets for Large inflation}
    \end{itemize}

    \section{Deriving Inequalities}
    \begin{itemize}
        \item \todo[TC]{Marginal problem}
        \item \todo[TC]{Popular methods: Fourier Motzkin (Convex hull, Polytope projection), Hardy implication inequalities, linear program/certificate}
        \item \todo[TC]{Overview incidence for Large Inflation}
        \item \todo[TC]{Rule out expensive methods like FM}
        \item \todo[TC]{Present some of the inequalities found}
    \end{itemize}

    \subsection{Symmetric Inequalities}
    \begin{itemize}
        \item \todo[TC]{Discuss symmetries and why they are useful}
        \item \todo[TC]{Symmetrizing Incidence Matrix}
        \item \todo[TC]{Large inflation incidence contracted drastically}
        \item \todo[TC]{Present some of the inequalities found}
    \end{itemize}

    \section{Violations}
    \begin{itemize}
        \item \todo[TC]{Fritz Distribution violates found inequalities}
    \end{itemize}
    \subsection{Numerical Optimizations}
    \begin{itemize}
        \item \todo[TC]{Generic idea}
        \item \todo[TC]{Optimization techniques used}
    \end{itemize}
    \subsection{Parameterizing Quantum Distributions}
    \begin{itemize}
        \item \todo[TC]{Unitaries}
        \item \todo[TC]{States}
        \item \todo[TC]{Measurements}
        \item \todo[TC]{Aligning States / Permutation}
    \end{itemize}
    \subsection{Results}
    \begin{itemize}
        \item \todo[TC]{Plots of various optimizations}
        \item \todo[TC]{Features of maximally violating distributions}
    \end{itemize}

    \section{Conclusions}
    \begin{itemize}
        \item \todo[TC]{Inflation technique capable of finding inequalities witnessing quantum/classic difference in TS}
        \item \todo[TC]{Causal incompatibility inequalities found violated by known distributions}
        \item \todo[TC]{Maximal distributions are different than Fritz but still rely on Bell's theorem}
        \item \todo[TC]{Refinement on Fritz's question}
    \end{itemize}

    \section*{Acknowledgments}
    \begin{itemize}
        \item \todo[TC]{Elie}
        \item \todo[TC]{Perimeter}
        \item \todo[TC]{University of Waterloo}
        \item \todo[TC]{Possible due to Mike Lazaridis Scholarship}
    \end{itemize}


\end{document}