\documentclass{article}
\usepackage{xcolor}

\usepackage[backend=biber, style=authoryear, doi=false,isbn=false,url=false, citestyle=authoryear-comp]{biblatex} %
\addbibresource{../references.bib}

\newcommand{\newslide}{\textcolor{red}{[New~Slide]}}
\newcommand{\elaborate}{\textcolor{gray}{[Elaborate]}}


\begin{document}
    Hello everyone, I'd like to begin by thanking the International Institute of Physics and the organizers of this workshop for inviting me to present.
    I regret not being able to attend in person and am sincerely gracious for being permitted to deliver my presentation via Skype.
    Moreover, I'd like to thank the Perimeter Institute in Canada for directly funding this research. \\

    Given that it is likely very few of you know who I am, I believe a short introduction is warranted. My name is TC Fraser and I am currently completing my undergraduate degree at the University of Waterloo in Canada. This research project was the result of me being awarded the Mike Lazaridis Scholarship for Theoretical Physics which enabled me to conduct research at the Perimeter Institute in Canada over the last several months. I am currently supervised by Elie Wolfe, and nearly all of the following results were made possible by his guidance and mentor-ship. \\

    The objective of my research was to study and characterize quantum non-locality in a very particular causal structure know as the triangle scenario. \newslide The triangle scenario consists of three observers (A,B,C) arranged in a triangular manner each pair-wise sharing local latent/hidden variables. Throughout this talk I will be representing observable nodes as blue triangles and latent nodes as orange circles. \\

    More specifically, the primal objective of this research was to derive causal compatibility inequalities for the triangle scenario that are violated by quantum correlations; something that nobody has obtained previous. The purpose of doing so is to search for novel non-local resources that are potentially different than those known previously. \\

    The triangle scenario has been studied previously:
    In~\cite{Branciard_2012} it was noted that characterizing locality in TS remained an open problem and that identifying compatibility constraints in this configuration \textbf{seemed challenging}.
    {In~\cite{Fritz_2012}, Fritz demonstrated that TS is the \textbf{smallest} correlation scenario in which their exists quantum incompatible distributions (proof without inequalities)}; an example I will elucidate later in this talk.
    {In~\cite{Henson_2014}, TS was classified as an \textbf{interesting} causal structure: conditional independence relations are not a sufficient characterization of compatibility (there are none)}.
    {Several other authors (see~\cite{Steudel_2010},~\cite{Chaves_2014},~\cite{Inflation}, $\ldots$) have investigated TS without achieving such inequalities.}

    \begin{center}
        \textbf{Part 1: Inflation Technique}
    \end{center}
    \newslide

    To that end, in this talk I will be summarizing a new tool that can be used to solve numerous causal inference and causal compatibility problems called the Inflation Technique developed by ~\cite{Inflation}. The Inflation Technique provides an avenue for constructing polynomial inequalities that constraint the set of compatible distributions. \\

    In order to demonstrate the typical procedures that comprise the inflation technique, I will work through a very simple example of non-local correlations on the triangle scenario. Doing so will motivate how I constructed compatibility inequalities that can be violated by quantum correlations. \\

    To conclude, I will discuss how these inequalities can be used to help search for novel incompatible quantum distributions using numerical optimization techniques. \\

    \newslide

    Before continuing, it is beneficial to define precisely what is meant by causal compatibility in the triangle scenario. A probability distribution over $A,B,C$ is compatible with the triangle scenario if it can be written as the product over the causal parameters of the triangle. That is to say, a joint distribution factorizes into distributions over each node given its parents in the triangle scenario. Effectively, compatibility corresponds to the probabilistic independence relations generated by the graphical structure of the triangle scenario. \\

    The more general setting for causal compatibility has two necessary components: the existence of a joint distribution over all nodes in the causal structure and that the joint distribution factors into causal parameters. \elaborate\\

    \newslide

    I will now introduce you to the core concepts and features that go into the Inflation Technique. It is important to remember that the inflation technique is a tool to solve numerous causal inference problems and can be applied to effectively any causal structure. Nonetheless I will be focusing on its application to the triangle scenario specifically. \\

    \newslide

    The inflation technique concerns itself with the following idea. Given a causal structure (such as the triangle scenario depicted of the left), one can construct an inflated causal structure such as the one depicted on the right. A inflation resembles the original causal structure in two ways:
    \begin{enumerate}
        \item The nodes in the inflated causal structure share the same labels as those in the original causal structure up to a numerical subscript called the copy-index. The copy-index simply is used to distinguish different copies of nodes from the original causal structure.
        \item The local ancestral features of the inflated causal structure are identical to those of the original causal structure.
    \end{enumerate}
    \elaborate\\

    \newslide\\
    To illustrate this concretely, examine node $A_1$ of the inflated causal structure.\\
    \elaborate\\
    \newslide\\
    \elaborate\\
    \newslide\\
    As you might imagine, inflations are not unique. Depicted here are a few inflations of the triangle scenario that have been of use when it comes to examining non-locality in the the triangle scenario. It is useful and quite amusing to invent names for the various inflations.
    \elaborate\\
    \newslide\\
    Now that we have an idea of inflations are, I want to introduce a very important type of a set of node.
    \elaborate
    \newslide\\
    Injectable sets are very useful because of the following inflation lemma that connects compatibility on the original causal structure to compatibility on the inflated causal structure.\\

    If one manages to obtain a compatible marginal model defined over the images of the injectable sets, then by definition that means there exists a set of causal parameters. From this, we can construct a set of inflated causal parameters defined over the nodes of the inflated causal structure by copying the conditional distributions whenever necessary. As a result, we can construct all sorts of compatible distributions on the inflated causal structure. \\

    By itself, the inflation lemma isn't very useful, but as a corollary, we can apply this process in reverse to inequalities on the inflated causal.
    \elaborate
    \newslide\\
    To demonstrate this procedure concretely, I will prove that perfect correlations are not compatible with the triangle scenario. It is important to remember that there are certainly more intuitive and immediate ways of demonstrating that perfect correlations are non-local, I only present the argument in this way to illustrate exactly how the more general procedure can be implemented to find compatibility inequalities that are violated by quantum distributions.
    \elaborate
    \elaborate
    \elaborate

    \begin{center}
        \textbf{Part 2: Quantum Non-locality From Inflation}
    \end{center}

    At this stage, it is my hope that you have a rough idea about how to generate polynomial compatibility inequalities using the inflation technique. Moreover it should be clear that if you have non-local distributions that are know to be non-local by other means, then one can manufacture an inequality that witnesses it's non-locality. \\

    With this guiding principle in mind, I can find inequalities that are violated by quantum distributions if I know of some non-local quantum distributions ahead of time. \\

    \elaborate

    \begin{center}
        \textbf{Part 2.5: Fritz Distribution}
    \end{center}

    I mentioned early that Tobias Fritz discovered quantum non-locality in the triangle scenario and proved its non-locality without explicitly using inequalities. The brilliant insight he made was to recognize that the famous Bell scenario can be effectively embedded into the triangle scenario in the following way. \\

    \newslide

    By simply rearranging/relabeling the nodes of the triangle scenario, one can recover something that looks remarkably similar to a Bell scenario between $A$ and $B$ where each party has a measurement setting $S_A$ and $S_B$ that determines the measurement to be made on their shared resource $\lambda$. \\

    The key difference between the Bell scenario and this embedding is that the measurement pusedo-settings are latent nodes. Therefore in order to directly mimic the Bell scenario, $C$ needs to taken on the role of announcing the measurement settings for $A$ and $B$. After doing so, any non-locality in the Bell scenario can be embedded into non-locality in of the triangle scenario. \\

    If $A,B,C$ are given $4$-outcome measurements, we can achieve Bell non-locality explicitly.

    \newslide
    \elaborate


    \printbibliography


\end{document}